% Created 2021-05-11 Tue 16:36
% Intended LaTeX compiler: pdflatex
\documentclass[11pt]{article}
\usepackage[utf8]{inputenc}
\usepackage[T1]{fontenc}
\usepackage{graphicx}
\usepackage{grffile}
\usepackage{longtable}
\usepackage{wrapfig}
\usepackage{rotating}
\usepackage[normalem]{ulem}
\usepackage{amsmath}
\usepackage{textcomp}
\usepackage{amssymb}
\bibliographystyle{elsarticle-num}
\usepackage{capt-of}
\usepackage{hyperref}
\date{\today}
\title{}
\hypersetup{
 pdfauthor={},
 pdftitle={},
 pdfkeywords={},
 pdfsubject={},
 pdfcreator={Emacs 27.2 (Org mode 9.4.5)}, 
 pdflang={English}}
\begin{document}

\tableofcontents

\section{Resumo}
\label{sec:org03048d0}
Derivou-se as leis de Ohm, a qual originalmente veio de uma série de proposições de proporções entre grandezas físicas. Primeiro, considera-se a relação entre Voltagem e Corrente, apenas. Em sequência, as relações entre Corrente e Resistividade. Assim, deduz-se a forma da relação de \(I(V,R)\).  Também, ascultamos a forma da Resistência, em função da da condutividade intrínseca do material e a sua forma geométrica - extrínseca - . i.e., $R(\rho, l, a)$. Assim, nossos experimentos demonstram como derivar as leis de Ohm, a partir de experimentos símiles ao feito pelo Georg Ohm.

\section{Introdução}
\label{sec:org0fba29c}
\subsection{Bases teóricas e deduções}
\label{sec:orgbcdb80b}
A lei de Ohm pode ser obtida a partir da variação parcial de parâmetros físicos. Se variarmos, para um dado material, e forma fixos, a Diferença de Potencial (\(V\)) entre duas pontas do material, então, necessariamente, há uma variação de Corrente (\(I\)).  De onde observa-se uma dependência diretamente proporcional entre \(I\) e \(V\).

\begin{equation}
\label{eq:IV}
I \propto V
\end{equation}

Ademais, ao variarmos o material, e fixarmos a geometria, temperatura e potencial, vemos uma mudança na Corrente \(I\). Seja \(\sigma\) a condutividade intrínseca do material,
\begin{equation}
\label{eq:IV}
I \propto \sigma
\end{equation}

Seja a resistividade intrínseca do material - \(\rho\) - o inverso da condutividade,

\begin{equation}
\label{eq:IV}
I \propto \frac{1}{\rho}
\end{equation}

Por fim, variando-se a geometria, e fixando-se todos os outros parâmetros, encontramos que

\begin{equation}
\label{eq:IV}
(I \propto A) \land (I \propto \frac{1}{l})
\end{equation}

Em que \(A\) é a área transversal à corrente \(I\). E, \(l\) a extensão do fio, paralelo a direção da corrente.

Dessa forma, deriva-se a lei de Ohm experimental,
\begin{equation}
  \begin{aligned}
  I &= \frac{\sigma A V}{l} \\
    \Leftrightarrow I &= \frac{V}{(\frac{\rho l}{A})}\\
    \Leftrightarrow I &= \frac{V}{R} \quad \land \quad R = \frac{\rho l}{A} 
  \end{aligned}
\end{equation}

\subsection{Implicações tecnológicas}
\label{sec:orgb9b70b1}
A escolha de materiais é de extrema importância, na confecção de produtos eletrônicos, visto que pode-se aumentar a potência de processamento  - causado pela corrente de elétrons - sem, necessariamente, ter um aumento de voltagem e consequênte consumo de energia. Trabalhando na escolha dos materiais e sua geometria, pode-se controlar uma série de parâmetros de performace de equipamentos eletrônicos.

\subsubsection{Lei de Moore}
\label{sec:org7257ec3}
De fato, essa lógica de aumento de potência, influênciando-se a forma e propriendades intrínsecas do material é tão forte que existe uma "Lei" empírica, a qual relaciona o aumento do número de resistores por unidade de área de um microchip. Consequentemente, essa também é uma "Lei" sobre a potência de processamento dos chips \cite{moore2012computing}.

\subsubsection{Barreiras e limites clássicos}
\label{sec:org3d1d2fb}
Quando se diminui o tamanho dos resistores em microchips, para aumentar sua eficiência, existe um limite para o comportamento esperado, pela lei de Ohm. Pois, os materiais se portam de maneira não linear, e não clássica. Pois, em realidade, a aproximação \textit{Clássica} não é suficiente para prever o comportamento do material \cite{moore2012computing}.

\subsubsection{Região não-Ohmica}
\label{sec:org251ff4b}
Ultrapassando-se o limite clássico, qualquer aumento de Corrente está ligada a um aumento exponencial da Voltagem. Assim, em limites não-Ohmicos, a lógica de aumentar eficiência, por meio da geometria do material, se torna insuficiente. E, este seria o limite da  "Lei" de Moore, prevista para ser atingida, comercialmente, em 2021-2022 \cite{forrest_2016}.

\section{Referências}
\bibliography{referencias}
\end{document}