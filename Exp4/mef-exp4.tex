% Created 2021-06-16 Wed 16:27
% Intended LaTeX compiler: pdflatex
\documentclass[11pt]{article}
\usepackage[utf8]{inputenc}
\usepackage[T1]{fontenc}
\usepackage{graphicx}
\usepackage{grffile}
\usepackage{longtable}
\usepackage{wrapfig}
\usepackage{rotating}
\usepackage[normalem]{ulem}
\usepackage{amsmath}
\usepackage{textcomp}
\usepackage{amssymb}
\usepackage{capt-of}
\usepackage{hyperref}
\usepackage{bibtex}

\date{\today}
\title{}
\hypersetup{
 pdfauthor={},
 pdftitle={},
 pdfkeywords={},
 pdfsubject={},
 pdfcreator={Emacs 27.2 (Org mode 9.4.6)}, 
 pdflang={English}}
\begin{document}

\tableofcontents

\section{Introdução}
\label{sec:orga917fff}
\subsection{Resistência}
\label{sec:org2295ef2}
A resistência pode ser vista tanto macroscopicamente, quando
microscopicamente. O modelo microscópico da resistividade nos
explana o porquê dos materiais intrinsicamente conduzirem como
metais, ou semi-materiais. A explicação se dá por meio dos Gaps de
condução eletrônica do material.

Dado que a resistividade pode ser
pensada como medida relacionada com o livre caminho médio de um
elétron, portadores de energia, quando menos empedidos, melhor sua
condução, e sua classificaçao se dá como metal. Naturalmente, as
estruturas microscópicas desses materiais não são células
periódias, as quais rigidamente contém os eletrons. Mas, como um
sistema de compartilhamento de eletrons livres - uma núvem
eletrônica \cite{schrodinger1935present}.

\subsection{\emph{Gaps} de energia}
\label{sec:orga315307}
A chance de se encontrar um eletron, microscopicamente, é dado pela
distribuição de Fermi-Dirac,

   \begin{equation}
f(E) = \dfrac{1}{e^{\left(\frac{E-E_F}{kT} \right)}+1}
   \end{equation}

Para os semi-condutores, \(E_f\) não está dentro da região de valores da
banda de condução. Assim, necessariamente, existe um \(\Delta E\), o
chamado \emph{Gap} de energia. Em geral, nos semi-condutores possuem
\(\textrm{GAP} > 2 \textrm{e.V.}\).

Para os condutores, esse GAP é próximo de zero. Mas, de forma
geral, \(\textrm{GAP} < 2 \textrm{e.V.}\).

\subsection{Linearização e Condutividade de semicondutores}
\label{sec:org2cf9c92}
A condutividade, provida por cargas e vacâncias, é dado por:
\(\sigma = n_i |e| (\mu_n + \mu_p)\).

Quando equacionamos esse valor, em termos do Gap de energia, temos:
\(\sigma = \sigma_0 e^{\frac{E_{\textrm{GAP}}}{2kT}}\).

Linearizando a equação, podemos obter:
\(E_{\textrm{gap}} = 2k.\ln{(\frac{\sigma}{\sigma0})}.T\)


\section{Materiais e Métodos}
\label{sec:org286d17a}
\subsection{Materiais}
\label{sec:orgec19cb2}
\subsubsection{Meçores de variações de medidas sistema}
\label{sec:orgb04b2e3}
Os aparatos experimentais de medição foram um termo resistor e dois
multímetros, MDM-8156B-1300-BR, Minipa; Keithley modelo 2000, o qual mede a
resistência do termoresistor de platina PT1000.

\subsubsection{Fontes de energia do sistema}
\label{sec:orgc20ba61}
Para gerar a diferença de potencial entre os terminais da placa,
empregou-se uma fonte Lakeshore série 100, modelo 120; fonte do termoresistor,
e uma fonte Politerm POL 16E, para gerar corrente.

\subsubsection{Aparatos utilitários}
\label{sec:org0b066b6}
Ademais, foram utilizado um paquímetro, Mitutoyo Universal, para obter medidas
de distância entre terminais e uma lixa para purificar a superfice
dos materiais. Por fim, utilizou-se do Nitrogênio para resfriar o
sistema e obter medidas variantes de temperatura.

\subsection{Métodos}
\label{sec:orgf6fd7f7}
Utilizou-se no experimento o método das quatro pontas. Ademais,
foram coletados dados de Cu, Fe, Nb e Si. Assim, obtivemos medidas
de Resistividade (\(\Omega\)) em função da Temperatura (\(K\)).

\subsubsection{Materias avaliados}
\label{sec:org3be70eb}
\begin{enumerate}
\item Coeficiente de resistividade (\(\alpha_T\))
\begin{itemize}
\item Cu, Fe, Nb, Si
\end{itemize}
\item Resistividade (\(\rho\))
\begin{itemize}
\item Cu, Fe, Nb, Si
\end{itemize}
\item Gap de energia
\begin{itemize}
\item Si
\end{itemize}
\end{enumerate}

\subsubsection{Os materiais das placas}
\label{sec:orgb54939c}
As duas placas metálicas constituia-se de:
\begin{itemize}
\item Fe
\item Si
\end{itemize}
\subsubsection{Os materiais do fio de condução.}
\label{sec:orgf1d6452}
Os fios eram compostos por:
\begin{itemize}
\item Nb
\item Cu
\end{itemize}

Os contatos foram soldados com finas camadas de prata.

\subsubsection{Métodos numérios}
\label{sec:org9708afa}
A determinação de erros foram medidos no processo de linearização,
para \(a\) e \(b\). Desta forma, calculculou-se o \(Gap\) de energia foi
calculado do Si.

Ademais, categorizou-se as curvas de \(\sigma\) e \(\rho\), por meio
da variação da temperatura. Por conseguinte, determinou-se os
comportamentos como condutores ou semi-condutores.

\bibliography{bib}
\end{document}